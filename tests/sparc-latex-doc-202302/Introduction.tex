\begin{frame}[allowframebreaks]{\textbf{Introduction}} \label{Introduction}
  SPARC is an open-source software package for the accurate, efficient, and scalable solution of the Kohn-Sham density functional theory (DFT) problem. The main features of SPARC currently include
  \begin{itemize}
    \item Applicable to isolated systems such as molecules as well as extended systems such as crystals, surfaces, and wires.
    \item Local, semilocal, and nonlocal (including hybrid) exchange-correlation functionals.
    \item Standard ONCV pseudopotentials, including nonlinear core corrections.
    \item Calculation of ground state energy, atomic forces, and stress tensor.
    \item Structural relaxation and ab initio molecular dynamics (NVE, NVT, and NPT).
    \item Spin polarized and unpolarized calculations.
    \item Spin-orbit coupling.
    \item Dispersion interactions through DFT-D3, vdW-DF1, and vdW-DF2.
  \end{itemize}
  % Additional details regarding the formulation and implementation of SPARC can be found in the accompanying paper. Please direct any questions and report any bugs to Prof.~Phanish Suryanarayana.
  
  \end{frame}
  
  
  %%%%%%%%%%%%%%%%%%%%%%%%%%%%%%%%%%%%%%%%%%%%%%%%%%
  \begin{frame}[allowframebreaks]{\textbf{Citation}} \label{Citation}
  If you publish work using/regarding SPARC, please cite some of the following articles, particularly those that are most relevant to your work:
  \begin{itemize}
      \item General: \url{https://doi.org/10.1016/j.softx.2021.100709}, \url{https://doi.org/10.1016/j.cpc.2016.09.020}, \url{https://doi.org/10.1016/j.cpc.2017.02.019}
      \item Non-orthogonal systems: \url{https://doi.org/10.1016/j.cplett.2018.04.018}
      \item Linear solvers: \url{https://doi.org/10.1016/j.cpc.2018.07.007},    \url{https://doi.org/10.1016/j.jcp.2015.11.018}
      \item Stress tensor/pressure: \url{https://doi.org/10.1063/1.5057355}
      \item Atomic forces: \url{https://doi.org/10.1016/j.cpc.2016.09.020}, \url{https://doi.org/10.1016/j.cpc.2017.02.019}
      \item Mixing: \url{https://doi.org/10.1016/j.cplett.2016.01.033}, \url{https://doi.org/10.1016/j.cplett.2015.06.029}, \url{https://doi.org/10.1016/j.cplett.2019.136983}
  \end{itemize}
  \end{frame}
  
  
  %%%%%%%%%%%%%%%%%%%%%%%%%%%%%%%%%%%%%%%%%%%%%%%%%%
  \begin{frame}[allowframebreaks]{\textbf{Acknowledgments}} \label{Acknowledgments}
  
  \begin{itemize}
      \item \textbf{U.S. Department of Energy (DOE), Office of Science (SC): DE-SC0019410} \\
      \item \textbf{U.S. Department of Energy (DOE), National Nuclear Security Administration (NNSA): Advanced Simulation and Computing (ASC) Program} \\
      \vspace{10pt}
      \begin{itemize}
          \item Preliminary developments
          \begin{itemize}
              \item U.S. National Science Foundation: 1553212, 1663244, and 1333500
          \end{itemize}
      \end{itemize}
      
      
  \end{itemize}
  
  \end{frame}
  
  \begin{frame}[allowframebreaks,fragile]{\textbf{Installation - Option 1}} \label{Installation:1}
  Prerequisite: C compiler, MPI.\\
  There are several options to compile SPARC, depending on the available external libraries.
  \begin{itemize}
  \item Option 1: Compile with BLAS and LAPACK.
    \begin{itemize}
      \item Step 1: Install/Load OpenBLAS/BLAS and LAPACK.
      \item Step 2: Go to \texttt{src/} directory, there is an available \texttt{makefile}.
      \item Step 3 (optional): Edit \texttt{makefile}. If the BLAS library path and LAPACK library path are not in the search path, edit the \texttt{BLASROOT} and \texttt{LAPACKROOT} variables, and add them to \texttt{LDFLAGS}. If you are using BLAS instead of OpenBLAS, replace all \texttt{-lopenblas} flags with \texttt{-lblas}.
      \item Step 4 (optional): To turn on \texttt{DEBUG} mode, set \texttt{DEBUG\_MODE} to $1$ in the \texttt{makefile}.
      \item Step 5: Within the \texttt{src/} directory, compile the code by \\
            \texttt{\$ make clean; make}
    \end{itemize}
  \end{itemize}
  \end{frame}
  
  \begin{frame}[allowframebreaks,fragile]{\textbf{Installation - Option 2}} \label{Installation:2}
  
  \begin{itemize}
  \item Option 2 (default): Compile with MKL.
    \begin{itemize}
      \item Step 1: Install/Load MKL.
      \item Step 2: Go to \texttt{src/} directory, there is an available \texttt{makefile}.
      \item Step 3: Edit \texttt{makefile}. Set \texttt{USE\_MKL} to $1$ to enable compilation with MKL. If the MKL library path is not in the search path, edit the \texttt{MKLROOT} variable to manually set the MKL path.
      \item Step 4 (optional): For the projection/subspace rotation step, to use SPARC routines for matrix data distribution rather than ScaLAPACK (through MKL), set \texttt{USE\_DP\_SUBEIG} to 1. We found on some machines this option is faster.
      \item Step 5 (optional): To turn on \texttt{DEBUG} mode, set \texttt{DEBUG\_MODE} to $1$ in the \texttt{makefile}.
      \item Step 6: Within the \texttt{src/} directory, compile the code by \\
            \texttt{\$ make clean; make}
    \end{itemize}
  \end{itemize}
  
  %\framebreak
  \end{frame}
  
  
  \begin{frame}[allowframebreaks,fragile]{\textbf{Installation - Option 3}} \label{Installation:3}
  \begin{itemize}
  \item Option 3: Compile with BLAS, LAPACK, and ScaLAPACK.
    \begin{itemize}
      \item Step 1: Install/Load OpenBLAS/BLAS, LAPACK, and ScaLAPACK.
      \item Step 2: Go to \texttt{src/} directory, there is an available \texttt{makefile}.
      \item Step 3: Edit \texttt{makefile}. Set \texttt{USE\_SCALAPACK} to $1$ to enable compilation with ScaLAPACK. If the BLAS library path, LAPACK library path, and/or ScaLAPACK library path are not in the search path, edit the \texttt{BLASROOT}, \texttt{LAPACKROOT}, and/or \texttt{SCALAPACKROOT} variables accordingly, and add them to \texttt{LDFLAGS}. If you are using BLAS instead of OpenBLAS, replace all \texttt{-lopenblas} flags with \texttt{-lblas}.
      \item Step 4 (optional): For the projection/subspace rotation step, to use SPARC routines for matrix data distribution rather than ScaLAPACK, set \texttt{USE\_DP\_SUBEIG} to 1. We found on some machines this option is faster.		
      \item Step 5 (optional): To turn on \texttt{DEBUG} mode, set \texttt{DEBUG\_MODE} to $1$ in the \texttt{makefile}.
      \item Step 6: Within the \texttt{src/} directory, compile the code by \\
            \texttt{\$ make clean; make}
    \end{itemize}
  \end{itemize}
  
  \end{frame}
  
  \begin{frame}[allowframebreaks,fragile]{\textbf{Installation - lib}} \label{Installation:lib}
  Once compilation is done, a binary named \texttt{sparc} will be created in the \texttt{lib/} directory.
  \end{frame}
  
  
  
  
  \begin{frame}[allowframebreaks]{\textbf{Input files}} \label{Inputfiles}
  The required input files to run a simulation with SPARC are
  \begin{itemize}
    \item ``.inpt" file -- User options and parameters.
    \item ``.ion" file -- Atomic information.
  \end{itemize}
  It is required that the ``.inpt" and ``.ion" files are located in the same directory and share the same name. A detailed description of the input options is provided in this document. Examples of input files can be found in the directory \texttt{SPARC/tests}. \\
  \hspace{3mm}In addition, SPARC requires pseudopotential files of psp8 format which can be generated by D. R. Hamann's open-source pseudopotential code \href{http://www.mat-simresearch.com/}{ONCVPSP}. A large number of accurate and efficient pseudopotentials are already provided within the package. For access to more pseudopotentials, the user is referred to the \href{http://www.quantum-simulation.org/potentials/sg15_oncv/}{SG15 ONCV potentials}. Using the \href{http://www.mat-simresearch.com/}{ONCVPSP} input files included in the \href{http://www.quantum-simulation.org/potentials/sg15_oncv/}{SG15 ONCV potentials}, one can easily convert the \href{http://www.quantum-simulation.org/potentials/sg15_oncv/}{SG15 ONCV potentials} from upf format to psp8 format. Paths to the pseudopotential files are specified in the ``.ion" file.
  \end{frame}
  
  
  
  \begin{frame}[allowframebreaks,fragile]{\textbf{Execution}} \label{Execution}
  SPARC can be executed in parallel using the \texttt{mpirun} command. Sample PBS script files are available in ``SPARC/tests" folder. It is required that the ``.inpt" and ``.ion" files are located in the same directory and share the same name. For example, to run a simulation with 8 processes with input files as ``filename.inpt" and ``filename.ion" in the root directory (\texttt{SPARC/}), use the following command:
  \begin{verbatim}
  $ mpirun -np 8 ./lib/sparc -name filename
  \end{verbatim} 
  As an example, one can run one of the tests located in `SPARC/tests/`. First go to `SPARC/tests/Example\_tests/` directory:
  \begin{verbatim}
  $ $ cd tests/Example_tests/
  \end{verbatim} 
  There are a few input files available. Run a DC silicon system by
  \begin{verbatim}
  $ mpirun -np 24 ../../lib/sparc -name Si8_kpt
  \end{verbatim} 
  
  The result is printed to output file ``Si8\_kpt.out", located in the same directory as the input files. If the file ``Si8\_kpt.out" is already present, the result will be printed to ``Si8\_kpt.out\_1" instead. The max number of ``.out" files allowed with the same name is 100. Once this number is reached, the result will instead overwrite the ``Si8\_kpt.out" file. One can compare the result with the reference out file named ``Si8\_kpt.refout".\\%\\~\\
  
  % The result is printed to output file ``Si8-ONCV-0.4.out", located in the same directory as the input files. If the file ``Si8-ONCV-0.4.out" is already present, the result will be printed to ``Si8-ONCV-0.4.out\_1" instead. The max number of ``.out" files allowed with the same name is 100. Once this number is reached, the result will instead overwrite the ``Si8-ONCV-0.4.out" file. One can compare the result with the reference out file named ``Si8-ONCV-0.4.refout".\\%\\~\\
  
  \hspace{3mm} In the \texttt{tests/} directory, we also provide a suite of tests which are arranged in a hierarchy of folders. Each test system has its own directory. A python script is also provided which launches the suite of test systems. To run a set of four quick tests locally on the CPU, simply run: 
  \begin{verbatim}
  $ python test.py quick_run
  \end{verbatim} 
  
  The result is stored in the corresponding directory of the tests. A message is also printed in the terminal showing if the tests passed or failed. The tests can also be launched in parallel on a cluster by using the Python script. Detailed information on using the python script can be found in the 'ReadMe' file in the `tests/` directory.
  \end{frame}
  
  
  \begin{frame}[allowframebreaks,fragile]{\textbf{Output}} \label{Output}
  Upon successful execution of the \texttt{sparc} code, depending on the calculations performed, some output files will be created in the same location as the input files. \\
  
  \textbf{Single point calculations}  \\
  \begin{itemize}
    \item ``.out" file -- General information about the test, including input parameters, SCF convergence progress, ground state properties and timing information.
    \item ``.static" file -- Atomic positions and atomic forces if the user chooses to print these information.
  \end{itemize}
  
  \textbf{Structural relaxation calculations}  \\
  \begin{itemize}
    \item ``.out" file -- See above.
    \item ``.geopt" file -- Atomic positions and atomic forces for atomic relaxation, cell lengths and stress tensor for volume relaxation, and atomic positions, atomic forces, cell dimensions, and stress tensor for full relaxation.
    \item ``.restart" file -- Information necessary to perform a restarted structural relaxation calculation. Only created if atomic relaxation is performed.
  \end{itemize}
  
  \textbf{Quantum molecular dynamics (QMD) calculations}  \\
  \begin{itemize}
    \item ``.out" file -- See above.
    \item ``.aimd" file -- Atomic positions, atomic velocities, atomic forces, electronic temperature, ionic temperature and total energy for each QMD step.
    \item ``.restart" file -- Information necessary to perform a restarted QMD calculation. 
  \end{itemize}
  
  \end{frame}
  
  
  
  \begin{frame}[allowframebreaks]{\textbf{Input file options}} \label{Index}
  \vspace{-2mm}
  \begin{block}{System}
  \hyperlink{CELL}{\texttt{CELL}} $\vert$
  \hyperlink{LATVEC_SCALE}{\texttt{LATVEC\_SCALE}} $\vert$
  \hyperlink{LATVEC}{\texttt{LATVEC}}  $\vert$
  \hyperlink{FD_GRID}{\texttt{FD\_GRID}} $\vert$
  \hyperlink{MESH_SPACING}{\texttt{MESH\_SPACING}} $\vert$
  \hyperlink{ECUT}{\texttt{ECUT}} $\vert$
  \hyperlink{BC}{\texttt{BC}} $\vert$
  \hyperlink{FD_ORDER}{\texttt{FD\_ORDER}} $\vert$
  \hyperlink{EXCHANGE_CORRELATION}{\texttt{EXCHANGE\_CORRELATION}} $\vert$
  \hyperlink{SPIN_TYP}{\texttt{SPIN\_TYP}} $\vert$
  \hyperlink{KPOINT_GRID}{\texttt{KPOINT\_GRID}} $\vert$
  \hyperlink{KPOINT_SHIFT}{\texttt{KPOINT\_SHIFT}} $\vert$
  \hyperlink{ELEC_TEMP_TYPE}{\texttt{ELEC\_TEMP\_TYPE}} $\vert$
  \hyperlink{ELEC_TEMP}{\texttt{ELEC\_TEMP}} $\vert$
  \hyperlink{SMEARING}{\texttt{SMEARING}} $\vert$
  \hyperlink{NSTATES}{\texttt{NSTATES}}     $\vert$
  \hyperlink{D3_FLAG}{\texttt{D3\_FLAG}} $\vert$
  \hyperlink{D3_RTHR}{\texttt{D3\_RTHR}} $\vert$
  \hyperlink{D3_CN_THR}{\texttt{D3\_CN\_THR}} $\vert$
  \hyperlink{EXX_RANGE_FOCK}{\texttt{EXX\_RANGE\_FOCK}} $\vert$ 
  \hyperlink{EXX_RANGE_PBE}{\texttt{EXX\_RANGE\_PBE}} $\vert$ 
  \hyperlink{ATOM_TYPE}{\texttt{ATOM\_TYPE}} $\vert$
  \hyperlink{PSEUDO_POT}{\texttt{PSEUDO\_POT}}  $\vert$
  \hyperlink{N_TYPE_ATOM}{\texttt{N\_TYPE\_ATOM}} $\vert$
  \hyperlink{COORD}{\texttt{COORD}} $\vert$
  \hyperlink{COORD_FRAC}{\texttt{COORD\_FRAC}} $\vert$
  \hyperlink{RELAX}{\texttt{RELAX}} $\vert$
  \hyperlink{SPIN}{\texttt{SPIN}} 
  \end{block}
  
  \begin{block}{SCF}
  \hyperlink{CHEB_DEGREE}{\texttt{CHEB\_DEGREE}} $\vert$
  \hyperlink{CHEFSI_BOUND_FLAG}{\texttt{CHEFSI\_BOUND\_FLAG}} $\vert$
  \hyperlink{RHO_TRIGGER}{\texttt{RHO\_TRIGGER}} $\vert$
  \hyperlink{MAXIT_SCF}{\texttt{MAXIT\_SCF}} $\vert$
  \hyperlink{MINIT_SCF}{\texttt{MINIT\_SCF}} $\vert$
  \hyperlink{TOL_SCF}{\texttt{TOL\_SCF}} $\vert$
  \hyperlink{SCF_FORCE_ACC}{\texttt{SCF\_FORCE\_ACC}} $\vert$
  \hyperlink{SCF_ENERGY_ACC}{\texttt{SCF\_ENERGY\_ACC}} $\vert$
  \hyperlink{TOL_LANCZOS}{\texttt{TOL\_LANCZOS}} $\vert$
  \hyperlink{MIXING_VARIABLE}{\texttt{MIXING\_VARIABLE}} $\vert$
  \hyperlink{MIXING_HISTORY}{\texttt{MIXING\_HISTORY}} $\vert$
  \hyperlink{MIXING_PARAMETER}{\texttt{MIXING\_PARAMETER}} $\vert$
  \hyperlink{MIXING_PARAMETER_SIMPLE}{\texttt{MIXING\_PARAMETER\_SIMPLE}} $\vert$
  \hyperlink{MIXING_PARAMETER_MAG}{\texttt{MIXING\_PARAMETER\_MAG}} $\vert$
  \hyperlink{MIXING_PARAMETER_SIMPLE_MAG}{\texttt{MIXING\_PARAMETER\_SIMPLE\_MAG}} $\vert$
  \hyperlink{PULAY_FREQUENCY}{\texttt{PULAY\_FREQUENCY}} $\vert$
  \hyperlink{PULAY_RESTART}{\texttt{PULAY\_RESTART}} $\vert$
  \hyperlink{MIXING_PRECOND}{\texttt{MIXING\_PRECOND}} $\vert$
  \hyperlink{MIXING_PRECOND_MAG}{\texttt{MIXING\_PRECOND\_MAG}} $\vert$
  \hyperlink{TOL_PRECOND}{\texttt{TOL\_PRECOND}} $\vert$
  \hyperlink{PRECOND_KERKER_KTF}{\texttt{PRECOND\_KERKER\_KTF}} $\vert$
  \hyperlink{PRECOND_KERKER_THRESH}{\texttt{PRECOND\_KERKER\_THRESH}} $\vert$
  \hyperlink{PRECOND_KERKER_KTF_MAG}{\texttt{PRECOND\_KERKER\_KTF\_MAG}} $\vert$
  \hyperlink{PRECOND_KERKER_THRESH_MAG}{\texttt{PRECOND\_KERKER\_THRESH\_MAG}} $\vert$
  \hyperlink{FIX_RAND}{\texttt{FIX\_RAND}} $\vert$ 
  \hyperlink{TOL_FOCK}{\texttt{TOL\_FOCK}} $\vert$ 
  \hyperlink{MAXIT_FOCK}{\texttt{MAXIT\_FOCK}} $\vert$ 
  \hyperlink{MINIT_FOCK}{\texttt{MINIT\_FOCK}} $\vert$ 
  \hyperlink{TOL_SCF_INIT}{\texttt{TOL\_SCF\_INIT}} $\vert$ 
  \hyperlink{ACE_FLAG}{\texttt{ACE\_FLAG}} $\vert$ 
  \hyperlink{EXX_METHOD}{\texttt{EXX\_METHOD}} $\vert$ 
  \hyperlink{EXX_MEM}{\texttt{EXX\_MEM}} $\vert$ 
  \hyperlink{EXX_FRAC}{\texttt{EXX\_FRAC}} $\vert$ 
  \hyperlink{EXX_ACE_VALENCE_STATES}{\texttt{EXX\_ACE\_VALENCE\_STATES}} $\vert$ 
  \hyperlink{EXX_DOWNSAMPLING}{\texttt{EXX\_DOWNSAMPLING}} $\vert$ 
  \hyperlink{EXX_DIVERGENCE}{\texttt{EXX\_DIVERGENCE}}
  \end{block}
  
  \vspace{-2mm}
  \begin{block}{Electrostatics}
  \hyperlink{TOL_POISSON}{\texttt{TOL\_POISSON}} $\vert$
  \hyperlink{MAXIT_POISSON}{\texttt{MAXIT\_POISSON}} $\vert$
  \hyperlink{TOL_PSEUDOCHARGE}{\texttt{TOL\_PSEUDOCHARGE}} $\vert$
  \hyperlink{REFERENCE_CUTOFF}{\texttt{REFERENCE\_CUTOFF}} 
  \end{block}
  
  \vspace{-2mm}
  \begin{block}{Stress calculation}
  \hyperlink{CALC_STRESS}{\texttt{CALC\_STRESS}} $\vert$
  \hyperlink{CALC_PRES}{\texttt{CALC\_PRES}}
  \end{block}
  \vspace{-2mm}
  
  \begin{block}{QMD}
  \hyperlink{MD_FLAG}{\texttt{MD\_FLAG}} $\vert$
  \hyperlink{MD_METHOD}{\texttt{MD\_METHOD}} $\vert$
  \hyperlink{MD_NSTEP}{\texttt{MD\_NSTEP}} $\vert$
  \hyperlink{MD_TIMESTEP}{\texttt{MD\_TIMESTEP}} $\vert$
  \hyperlink{ION_TEMP}{\texttt{ION\_TEMP}} $\vert$
  \hyperlink{ION_TEMP_END}{\texttt{ION\_TEMP\_END}} $\vert$
  \hyperlink{ION_VEL_DSTR}{\texttt{ION\_VEL\_DSTR}} $\vert$
  \hyperlink{ION_VEL_DSTR_RAND}{\texttt{ION\_VEL\_DSTR\_RAND}} $\vert$
  \hyperlink{QMASS}{\texttt{QMASS}} $\vert$
  \hyperlink{NPT_NH_QMASS}{\texttt{NPT\_NH\_QMASS}} $\vert$
  \hyperlink{NPT_NH_BMASS}{\texttt{NPT\_NH\_BMASS}} $\vert$
  \hyperlink{NPT_NP_QMASS}{\texttt{NPT\_NP\_QMASS}} $\vert$
  \hyperlink{NPT_NP_BMASS}{\texttt{NPT\_NP\_BMASS}} $\vert$
  \hyperlink{NPT_SCALE_VECS}{\texttt{NPT\_SCALE\_VECS}} $\vert$
  \hyperlink{TARGET_PRESSURE}{\texttt{TARGET\_PRESSURE}} $\vert$
  \hyperlink{RESTART_FLAG}{\texttt{RESTART\_FLAG}} $\vert$
  \hyperlink{TWTIME}{\texttt{TWTIME}}
  \end{block}
  
  \vspace{-2mm}
  \begin{block}{Structural relaxation}
  \hyperlink{RELAX_FLAG}{\texttt{RELAX\_FLAG}} $\vert$
  \hyperlink{RELAX_METHOD}{\texttt{RELAX\_METHOD}} $\vert$
  \hyperlink{RELAX_NITER}{\texttt{RELAX\_NITER}} $\vert$
  \hyperlink{TOL_RELAX}{\texttt{TOL\_RELAX}} $\vert$
  \hyperlink{TOL_RELAX_CELL}{\texttt{TOL\_RELAX\_CELL}} $\vert$
  \hyperlink{RELAX_MAXDILAT}{\texttt{RELAX\_MAXDILAT}} $\vert$
  \hyperlink{NLCG_SIGMA}{\texttt{NLCG\_SIGMA}} $\vert$
  \hyperlink{L_HISTORY}{\texttt{L\_HISTORY}} $\vert$
  \hyperlink{L_FINIT_STP}{\texttt{L\_FINIT\_STP}} $\vert$
  \hyperlink{L_MAXMOV}{\texttt{L\_MAXMOV}} $\vert$
  \hyperlink{L_AUTOSCALE}{\texttt{L\_AUTOSCALE}} $\vert$
  \hyperlink{L_LINEOPT}{\texttt{L\_LINEOPT}} $\vert$
  \hyperlink{L_ICURV}{\texttt{L\_ICURV}} $\vert$
  \hyperlink{FIRE_DT}{\texttt{FIRE\_DT}} $\vert$
  \hyperlink{FIRE_MASS}{\texttt{FIRE\_MASS}} $\vert$
  \hyperlink{FIRE_MAXMOV}{\texttt{FIRE\_MAXMOV}} $\vert$
  \hyperlink{RESTART_FLAG}{\texttt{RESTART\_FLAG}}
  \end{block}
  
  \begin{block}{Print options}
  \hyperlink{PRINT_ATOMS}{\texttt{PRINT\_ATOMS}} $\vert$
  \hyperlink{PRINT_FORCES}{\texttt{PRINT\_FORCES}} $\vert$
  \hyperlink{PRINT_MDOUT}{\texttt{PRINT\_MDOUT}} $\vert$
  \hyperlink{PRINT_RELAXOUT}{\texttt{PRINT\_RELAXOUT}} $\vert$
  \hyperlink{PRINT_RESTART}{\texttt{PRINT\_RESTART}} $\vert$
  \hyperlink{PRINT_RESTART_FQ}{\texttt{PRINT\_RESTART\_FQ}} $\vert$
  \hyperlink{PRINT_VELS}{\texttt{PRINT\_VELS}} $\vert$
  \hyperlink{OUTPUT_FILE}{\texttt{OUTPUT\_FILE}} $\vert$
  \hyperlink{PRINT_EIGEN}{\texttt{PRINT\_EIGEN}} $\vert$
  \hyperlink{PRINT_DENSITY}{\texttt{PRINT\_DENSITY}} $\vert$
  \hyperlink{PRINT_ORBITAL}{\texttt{PRINT\_ORBITAL}} $\vert$
  \hyperlink{PRINT_ENERGY_DENSITY}{\texttt{PRINT\_ENERGY\_DENSITY}}
  \end{block}
  
  \begin{block}{Parallelization options}
  \hyperlink{NP_SPIN_PARAL}{\texttt{NP\_SPIN\_PARAL}} $\vert$
  \hyperlink{NP_KPOINT_PARAL}{\texttt{NP\_KPOINT\_PARAL}} $\vert$
  \hyperlink{NP_BAND_PARAL}{\texttt{NP\_BAND\_PARAL}} $\vert$
  \hyperlink{NP_DOMAIN_PARAL}{\texttt{NP\_DOMAIN\_PARAL}} $\vert$
  \hyperlink{NP_DOMAIN_PHI_PARAL}{\texttt{NP\_DOMAIN\_PHI\_PARAL}} $\vert$
  \hyperlink{EIG_SERIAL_MAXNS}{\texttt{EIG\_SERIAL\_MAXNS}} $\vert$
  \hyperlink{EIG_PARAL_BLKSZ}{\texttt{EIG\_PARAL\_BLKSZ}} $\vert$
  \hyperlink{EIG_PARAL_ORFAC}{\texttt{EIG\_PARAL\_ORFAC}} $\vert$
  \hyperlink{EIG_PARAL_MAXNP}{\texttt{EIG\_PARAL\_MAXNP}}
  \end{block}

  \end{frame}
  