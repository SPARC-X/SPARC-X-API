%%%%%%%%%%%%%%%%%%%%%%%%%%%%%%%%%%%%%%%%%%%%%%%%%%%%%%%%%%%%%%%%%%%%%%%%%%%%%%%%%%%%%%%%%%%%%
\begin{frame}[allowframebreaks,c]{} \label{Paral options}

\begin{center}
\Huge \textbf{Parallelization options}
\end{center}

\end{frame}
%%%%%%%%%%%%%%%%%%%%%%%%%%%%%%%%%%%%%%%%%%%%%%%%%%%%%%%%%%%%%%%%%%%%%%%%%%%%%%%%%%%%%%%%%%%%%


%%%%%%%%%%%%%%%%%%%%%%%%%%%%%%%%%%%%%%%%%%%%%%%%%%%%%%%%%%%%%%%%%%%%%%%%%%%%%%%%%%%%%%%%%%%%%
\begin{frame}[allowframebreaks]{\texttt{NP\_SPIN\_PARAL}} \label{NP_SPIN_PARAL}
\vspace*{-12pt}
\begin{columns}
\column{0.4\linewidth}
\begin{block}{Type}
Integer
\end{block}

\begin{block}{Default}
Automatically optimized
\end{block}

\column{0.4\linewidth}
\begin{block}{Unit}
No unit
\end{block}

\begin{block}{Example}
\texttt{NP\_SPIN\_PARAL}: 2
\end{block}
\end{columns}

\begin{block}{Description}
Number of spin groups.
\end{block}

\begin{block}{Remark}
This option is for development purpose. It's better to let SPARC choose the parallization parameters in practice.
\end{block}
\end{frame}
%%%%%%%%%%%%%%%%%%%%%%%%%%%%%%%%%%%%%%%%%%%%%%%%%%%%%%%%%%%%%%%%%%%%%%%%%%%%%%%%%%%%%%%%%%%%%


%%%%%%%%%%%%%%%%%%%%%%%%%%%%%%%%%%%%%%%%%%%%%%%%%%%%%%%%%%%%%%%%%%%%%%%%%%%%%%%%%%%%%%%%%%%%%
\begin{frame}[allowframebreaks]{\texttt{NP\_KPOINT\_PARAL}} \label{NP_KPOINT_PARAL}
\vspace*{-12pt}
\begin{columns}
\column{0.4\linewidth}
\begin{block}{Type}
Integer
\end{block}

\begin{block}{Default}
Automatically optimized
\end{block}

\column{0.4\linewidth}
\begin{block}{Unit}
No unit
\end{block}

\begin{block}{Example}
\texttt{NP\_KPOINT\_PARAL}: 5
\end{block}
\end{columns}

\begin{block}{Description}
Number of k-point groups.
\end{block}

\begin{block}{Remark}
This option is for development purpose. It's better to let SPARC choose the parallization parameters in practice.
\end{block}
\end{frame}
%%%%%%%%%%%%%%%%%%%%%%%%%%%%%%%%%%%%%%%%%%%%%%%%%%%%%%%%%%%%%%%%%%%%%%%%%%%%%%%%%%%%%%%%%%%%%


%%%%%%%%%%%%%%%%%%%%%%%%%%%%%%%%%%%%%%%%%%%%%%%%%%%%%%%%%%%%%%%%%%%%%%%%%%%%%%%%%%%%%%%%%%%%%
\begin{frame}[allowframebreaks]{\texttt{NP\_BAND\_PARAL}} \label{NP_BAND_PARAL}
\vspace*{-12pt}
\begin{columns}
\column{0.4\linewidth}
\begin{block}{Type}
Integer
\end{block}

\begin{block}{Default}
Automatically optimized
\end{block}

\column{0.4\linewidth}
\begin{block}{Unit}
No unit
\end{block}

\begin{block}{Example}
\texttt{NP\_BAND\_PARAL}: 5
\end{block}
\end{columns}

\begin{block}{Description}
Number of band groups.
\end{block}

\begin{block}{Remark}
This option is for development purpose. It's better to let SPARC choose the parallization parameters in practice.
\end{block}
\end{frame}
%%%%%%%%%%%%%%%%%%%%%%%%%%%%%%%%%%%%%%%%%%%%%%%%%%%%%%%%%%%%%%%%%%%%%%%%%%%%%%%%%%%%%%%%%%%%%


%%%%%%%%%%%%%%%%%%%%%%%%%%%%%%%%%%%%%%%%%%%%%%%%%%%%%%%%%%%%%%%%%%%%%%%%%%%%%%%%%%%%%%%%%%%%%
\begin{frame}[allowframebreaks]{\texttt{NP\_DOMAIN\_PARAL}} \label{NP_DOMAIN_PARAL}
\vspace*{-12pt}
\begin{columns}
\column{0.4\linewidth}
\begin{block}{Type}
Integer
\end{block}

\begin{block}{Default}
Automatically optimized
\end{block}

\column{0.4\linewidth}
\begin{block}{Unit}
No unit
\end{block}

\begin{block}{Example}
\texttt{NP\_DOMAIN\_PARAL}: 3 3 2
\end{block}
\end{columns}

\begin{block}{Description}
Dimensions of the 3D Cartesian topology embedded in each band group.
\end{block}

\begin{block}{Remark}
This option is for development purpose. It's better to let SPARC choose the parallization parameters in practice.
\end{block}
\end{frame}
%%%%%%%%%%%%%%%%%%%%%%%%%%%%%%%%%%%%%%%%%%%%%%%%%%%%%%%%%%%%%%%%%%%%%%%%%%%%%%%%%%%%%%%%%%%%%


%%%%%%%%%%%%%%%%%%%%%%%%%%%%%%%%%%%%%%%%%%%%%%%%%%%%%%%%%%%%%%%%%%%%%%%%%%%%%%%%%%%%%%%%%%%%%
\begin{frame}[allowframebreaks]{\texttt{NP\_DOMAIN\_PHI\_PARAL}} \label{NP_DOMAIN_PHI_PARAL}
\vspace*{-12pt}
\begin{columns}
\column{0.4\linewidth}
\begin{block}{Type}
Integer
\end{block}

\begin{block}{Default}
Automatically optimized
\end{block}

\column{0.4\linewidth}
\begin{block}{Unit}
No unit
\end{block}

\begin{block}{Example}
\texttt{NP\_DOMAIN\_PHI\_PARAL}: 1 1 2
\end{block}
\end{columns}

\begin{block}{Description}
Dimensions of the 3D Cartesian topology embedded in the global communicator.
\end{block}

\begin{block}{Remark}
This option is for development purpose. It's better to let SPARC choose the parallization parameters in practice.
\end{block}
\end{frame}
%%%%%%%%%%%%%%%%%%%%%%%%%%%%%%%%%%%%%%%%%%%%%%%%%%%%%%%%%%%%%%%%%%%%%%%%%%%%%%%%%%%%%%%%%%%%%


%%%%%%%%%%%%%%%%%%%%%%%%%%%%%%%%%%%%%%%%%%%%%%%%%%%%%%%%%%%%%%%%%%%%%%%%%%%%%%%%%%%%%%%%%%%%%
\begin{frame}[allowframebreaks]{\texttt{EIG\_SERIAL\_MAXNS}} \label{EIG_SERIAL_MAXNS}
\vspace*{-12pt}
\begin{columns}
\column{0.4\linewidth}
\begin{block}{Type}
Integer
\end{block}

\begin{block}{Default}
2000
\end{block}

\column{0.4\linewidth}
\begin{block}{Unit}
No unit
\end{block}

\begin{block}{Example}
\texttt{EIG\_SERIAL\_MAXNS}: 1000
\end{block}
\end{columns}

\begin{block}{Description}
Maximum \hyperlink{NSTATES}{\texttt{NSTATES}} value up to which a serial algorithm will be used to solve the subspace eigenproblem.
\end{block}

\begin{block}{Remark}
If one wants to use a parallel algorithm to solve the subspace eigenproblem for all cases, simply set \texttt{EIG\_SERIAL\_MAXNS} to $0$. Alternatively, set \texttt{EIG\_SERIAL\_MAXNS} to a very large value to always use serial algorithm.
\end{block}
\end{frame}
%%%%%%%%%%%%%%%%%%%%%%%%%%%%%%%%%%%%%%%%%%%%%%%%%%%%%%%%%%%%%%%%%%%%%%%%%%%%%%%%%%%%%%%%%%%%%


%%%%%%%%%%%%%%%%%%%%%%%%%%%%%%%%%%%%%%%%%%%%%%%%%%%%%%%%%%%%%%%%%%%%%%%%%%%%%%%%%%%%%%%%%%%%%
\begin{frame}[allowframebreaks]{\texttt{EIG\_PARAL\_BLKSZ}} \label{EIG_PARAL_BLKSZ}
\vspace*{-12pt}
\begin{columns}
\column{0.4\linewidth}
\begin{block}{Type}
Integer
\end{block}

\begin{block}{Default}
128
\end{block}

\column{0.4\linewidth}
\begin{block}{Unit}
No unit
\end{block}

\begin{block}{Example}
\texttt{EIG\_PARAL\_BLKSZ}: 64
\end{block}
\end{columns}

\begin{block}{Description}
Block size for the distribution of matrix in block-cyclic format in a parallel algorithm for solving the subspace eigenproblem.
\end{block}

\end{frame}
%%%%%%%%%%%%%%%%%%%%%%%%%%%%%%%%%%%%%%%%%%%%%%%%%%%%%%%%%%%%%%%%%%%%%%%%%%%%%%%%%%%%%%%%%%%%%



%%%%%%%%%%%%%%%%%%%%%%%%%%%%%%%%%%%%%%%%%%%%%%%%%%%%%%%%%%%%%%%%%%%%%%%%%%%%%%%%%%%%%%%%%%%%%
\begin{frame}[allowframebreaks]{\texttt{EIG\_PARAL\_ORFAC}} \label{EIG_PARAL_ORFAC}
\vspace*{-12pt}
\begin{columns}
\column{0.4\linewidth}
\begin{block}{Type}
Double
\end{block}
\begin{block}{Default}
0.0
\end{block}
\column{0.4\linewidth}
\begin{block}{Unit}
No unit
\end{block}
\begin{block}{Example}
\texttt{EIG\_PARAL\_ORFAC}: 0.001
\end{block}
\end{columns}
\begin{block}{Description}
Specifies which eigenvectors should be reorthogonalized when using the parallel eigensolver 
\href{https://software.intel.com/content/www/us/en/develop/documentation/onemkl-developer-reference-c/top/scalapack-routines/scalapack-driver-routines/p-syevx.html}{\texttt{p?syevx}} 
or 
\href{https://software.intel.com/content/www/us/en/develop/documentation/onemkl-developer-reference-c/top/scalapack-routines/scalapack-driver-routines/p-sygvx.html}{\texttt{p?sygvx}} 
for solving the subspace eigenproblem. 
The parallel eigensolvers can be turned on using the \hyperlink{EIG_SERIAL_MAXNS}{\texttt{EIG\_SERIAL\_MAXNS}} flag.
No reorthogonalization will be done if \texttt{EIG\_PARAL\_ORFAC} equals zero. A default value of $0.001$ is used if \texttt{EIG\_PARAL\_ORFAC} is negative. 
Note that reorthogonalization of eigenvectors is extremely time-consuming.
% See the corresponding documentation for more details.
\end{block}

\end{frame}
%%%%%%%%%%%%%%%%%%%%%%%%%%%%%%%%%%%%%%%%%%%%%%%%%%%%%%%%%%%%%%%%%%%%%%%%%%%%%%%%%%%%%%%%%%%%%


%%%%%%%%%%%%%%%%%%%%%%%%%%%%%%%%%%%%%%%%%%%%%%%%%%%%%%%%%%%%%%%%%%%%%%%%%%%%%%%%%%%%%%%%%%%%%
\begin{frame}[allowframebreaks]{\texttt{EIG\_PARAL\_MAXNP}} \label{EIG_PARAL_MAXNP}
\vspace*{-12pt}
\begin{columns}
\column{0.4\linewidth}
\begin{block}{Type}
Integer
\end{block}

\begin{block}{Default}
Generated by a linear model.
\end{block}

\column{0.4\linewidth}
\begin{block}{Unit}
No unit
\end{block}

\begin{block}{Example}
\texttt{EIG\_PARAL\_MAXNP}: 36
\end{block}
\end{columns}

\begin{block}{Description}
Maximum number of processors used in parallel eigensolver. The number is machine dependent. Users could provide their own value for best performance. 
\end{block}

\end{frame}
%%%%%%%%%%%%%%%%%%%%%%%%%%%%%%%%%%%%%%%%%%%%%%%%%%%%%%%%%%%%%%%%%%%%%%%%%%%%%%%%%%%%%%%%%%%%%
