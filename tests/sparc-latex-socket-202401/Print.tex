%%%%%%%%%%%%%%%%%%%%%%%%%%%%%%%%%%%%%%%%%%%%%%%%%%%%%%%%%%%%%%%%%%%%%%%%%%%%%%%%%%%%%%%%%%%%%
\begin{frame}[allowframebreaks,c]{} \label{Print options}

\begin{center}
\Huge \textbf{Print options}
\end{center}

\end{frame}
%%%%%%%%%%%%%%%%%%%%%%%%%%%%%%%%%%%%%%%%%%%%%%%%%%%%%%%%%%%%%%%%%%%%%%%%%%%%%%%%%%%%%%%%%%%%%




%%%%%%%%%%%%%%%%%%%%%%%%%%%%%%%%%%%%%%%%%%%%%%%%%%%%%%%%%%%%%%%%%%%%%%%%%%%%%%%%%%%%%%%%%%%%%
\begin{frame}[allowframebreaks]{\texttt{PRINT\_ATOMS}} \label{PRINT_ATOMS}
\vspace*{-12pt}
\begin{columns}
\column{0.4\linewidth}
\begin{block}{Type}
0 or 1
\end{block}

\begin{block}{Default}
0
\end{block}

\column{0.4\linewidth}
\begin{block}{Unit}
No unit
\end{block}

\begin{block}{Example}
\texttt{PRINT\_ATOMS}: 1
\end{block}
\end{columns}

\begin{block}{Description}
Flag for writing the atomic positions. For ground-state calculations, atom positions are printed to a `.static' output file. For structural relaxation calculations, atom positions are printed to a `.geopt' file. For QMD calculations, atom positions are printed to a `.aimd' file.
\end{block}


\end{frame}
%%%%%%%%%%%%%%%%%%%%%%%%%%%%%%%%%%%%%%%%%%%%%%%%%%%%%%%%%%%%%%%%%%%%%%%%%%%%%%%%%%%%%%%%%%%%%




%%%%%%%%%%%%%%%%%%%%%%%%%%%%%%%%%%%%%%%%%%%%%%%%%%%%%%%%%%%%%%%%%%%%%%%%%%%%%%%%%%%%%%%%%%%%%
\begin{frame}[allowframebreaks]{\texttt{PRINT\_FORCES}} \label{PRINT_FORCES}
\vspace*{-12pt}
\begin{columns}
\column{0.4\linewidth}
\begin{block}{Type}
0 or 1
\end{block}

\begin{block}{Default}
0
\end{block}

\column{0.4\linewidth}
\begin{block}{Unit}
No unit
\end{block}

\begin{block}{Example}
\texttt{PRINT\_FORCES}: 1
\end{block}
\end{columns}

\begin{block}{Description}
Flag for writing the atomic forces. For ground-state calculations, forces are printed to a `.static' output file. For structural relaxation calculations, forces are printed to a `.geopt' file. For QMD calculations, forces are printed to a `.aimd' file.
\end{block}

\end{frame}
%%%%%%%%%%%%%%%%%%%%%%%%%%%%%%%%%%%%%%%%%%%%%%%%%%%%%%%%%%%%%%%%%%%%%%%%%%%%%%%%%%%%%%%%%%%%%




%%%%%%%%%%%%%%%%%%%%%%%%%%%%%%%%%%%%%%%%%%%%%%%%%%%%%%%%%%%%%%%%%%%%%%%%%%%%%%%%%%%%%%%%%%%%%
\begin{frame}[allowframebreaks]{\texttt{PRINT\_MDOUT}} \label{PRINT_MDOUT}
\vspace*{-12pt}
\begin{columns}
\column{0.4\linewidth}
\begin{block}{Type}
0 or 1
\end{block}

\begin{block}{Default}
1
\end{block}

\column{0.4\linewidth}
\begin{block}{Unit}
No unit
\end{block}

\begin{block}{Example}
\texttt{PRINT\_MDOUT}: 0
\end{block}
\end{columns}

\begin{block}{Description}
Flag for printing the the QMD output into the .aimd file. 
\end{block}

%\begin{block}{Remark}
%\end{block}

\end{frame}
%%%%%%%%%%%%%%%%%%%%%%%%%%%%%%%%%%%%%%%%%%%%%%%%%%%%%%%%%%%%%%%%%%%%%%%%%%%%%%%%%%%%%%%%%%%%%





%%%%%%%%%%%%%%%%%%%%%%%%%%%%%%%%%%%%%%%%%%%%%%%%%%%%%%%%%%%%%%%%%%%%%%%%%%%%%%%%%%%%%%%%%%%%%
\begin{frame}[allowframebreaks]{\texttt{PRINT\_RELAXOUT}} \label{PRINT_RELAXOUT}
\vspace*{-12pt}
\begin{columns}
\column{0.4\linewidth}
\begin{block}{Type}
0 or 1
\end{block}

\begin{block}{Default}
1
\end{block}

\column{0.4\linewidth}
\begin{block}{Unit}
No unit
\end{block}

\begin{block}{Example}
\texttt{PRINT\_RELAXOUT}: 0
\end{block}
\end{columns}

\begin{block}{Description}
Flag for printing the structural relaxation data in a .geopt file.
\end{block}

\begin{block}{Remark}
Required only if \hyperlink{RELAX_FLAG}{\texttt{RELAX\_FLAG}} is greater than $0$.
\end{block}

\end{frame}
%%%%%%%%%%%%%%%%%%%%%%%%%%%%%%%%%%%%%%%%%%%%%%%%%%%%%%%%%%%%%%%%%%%%%%%%%%%%%%%%%%%%%%%%%%%%%



%%%%%%%%%%%%%%%%%%%%%%%%%%%%%%%%%%%%%%%%%%%%%%%%%%%%%%%%%%%%%%%%%%%%%%%%%%%%%%%%%%%%%%%%%%%%%
\begin{frame}[allowframebreaks]{\texttt{PRINT\_RESTART}} \label{PRINT_RESTART}
\vspace*{-12pt}
\begin{columns}
\column{0.4\linewidth}
\begin{block}{Type}
0 or 1
\end{block}

\begin{block}{Default}
1
\end{block}

\column{0.4\linewidth}
\begin{block}{Unit}
No unit
\end{block}

\begin{block}{Example}
\texttt{PRINT\_RESTART}: 0
\end{block}
\end{columns}

\begin{block}{Description}
Flag for writing the .restart file, used to restart QMD and structural relaxation simulations. 
\end{block}

\begin{block}{Remark}
Relevant only if either \hyperlink{MD_FLAG}{\texttt{MD\_FLAG}} is $1$ or \hyperlink{RELAX_FLAG}{\texttt{RELAX\_FLAG}} is $1$.
\end{block}

\end{frame}
%%%%%%%%%%%%%%%%%%%%%%%%%%%%%%%%%%%%%%%%%%%%%%%%%%%%%%%%%%%%%%%%%%%%%%%%%%%%%%%%%%%%%%%%%%%%%




%%%%%%%%%%%%%%%%%%%%%%%%%%%%%%%%%%%%%%%%%%%%%%%%%%%%%%%%%%%%%%%%%%%%%%%%%%%%%%%%%%%%%%%%%%%%%
\begin{frame}[allowframebreaks]{\texttt{PRINT\_RESTART\_FQ}} \label{PRINT_RESTART_FQ}
\vspace*{-12pt}
\begin{columns}
\column{0.4\linewidth}
\begin{block}{Type}
Integer
\end{block}

\begin{block}{Default}
1
\end{block}

\column{0.4\linewidth}
\begin{block}{Unit}
No unit
\end{block}

\begin{block}{Example}
\texttt{PRINT\_RESTART\_FQ}: 10
\end{block}
\end{columns}

\begin{block}{Description}
Frequency at which .restart file is written in QMD and structural optimization simulations.
\end{block}

\begin{block}{Remark}
Relevant only if either \hyperlink{MD_FLAG}{\texttt{MD\_FLAG}} is $1$ or \hyperlink{RELAX_FLAG}{\texttt{RELAX\_FLAG}} is $1$.
\end{block}

\end{frame}
%%%%%%%%%%%%%%%%%%%%%%%%%%%%%%%%%%%%%%%%%%%%%%%%%%%%%%%%%%%%%%%%%%%%%%%%%%%%%%%%%%%%%%%%%%%%%




%%%%%%%%%%%%%%%%%%%%%%%%%%%%%%%%%%%%%%%%%%%%%%%%%%%%%%%%%%%%%%%%%%%%%%%%%%%%%%%%%%%%%%%%%%%%%
\begin{frame}[allowframebreaks]{\texttt{PRINT\_VELS}} \label{PRINT_VELS}
\vspace*{-12pt}
\begin{columns}
\column{0.4\linewidth}
\begin{block}{Type}
0 or 1
\end{block}

\begin{block}{Default}
1
\end{block}

\column{0.4\linewidth}
\begin{block}{Unit}
No unit
\end{block}

\begin{block}{Example}
\texttt{PRINT\_VELS}: 0
\end{block}
\end{columns}

\begin{block}{Description}
Flag for printing the ion velocities in an QMD simulation into the .aimd file. 
\end{block}

\begin{block}{Remark}
Relevant only if \hyperlink{MD_FLAG}{\texttt{MD\_FLAG}} is set to $1$.
\end{block}

\end{frame}
%%%%%%%%%%%%%%%%%%%%%%%%%%%%%%%%%%%%%%%%%%%%%%%%%%%%%%%%%%%%%%%%%%%%%%%%%%%%%%%%%%%%%%%%%%%%%




%%%%%%%%%%%%%%%%%%%%%%%%%%%%%%%%%%%%%%%%%%%%%%%%%%%%%%%%%%%%%%%%%%%%%%%%%%%%%%%%%%%%%%%%%%%%%
\begin{frame}[allowframebreaks]{\texttt{OUTPUT\_FILE}} \label{OUTPUT_FILE}
\vspace*{-12pt}
\begin{columns}
\column{0.4\linewidth}
\begin{block}{Type}
String
\end{block}

\begin{block}{Default}
Same as the input file name
\end{block}

\column{0.4\linewidth}
\begin{block}{Unit}
No unit
\end{block}

\begin{block}{Example}
\texttt{OUTPUT\_FILE}: myfname
\end{block}
\end{columns}

\begin{block}{Description}
The name of the output files. The output files are attached with a suffix (`.out',`.static',`.geopt' and `.aimd'). 
\end{block}

\begin{block}{Remark}
If an output file with the same name already exist, the results will be written to a file with a number attached, e.g., `myfname.out\_1'. The maximum number of output files with the same name allowed is 100. After that the output files will be overwritten in succession.
\end{block}

\end{frame}
%%%%%%%%%%%%%%%%%%%%%%%%%%%%%%%%%%%%%%%%%%%%%%%%%%%%%%%%%%%%%%%%%%%%%%%%%%%%%%%%%%%%%%%%%%%%%



%%%%%%%%%%%%%%%%%%%%%%%%%%%%%%%%%%%%%%%%%%%%%%%%%%%%%%%%%%%%%%%%%%%%%%%%%%%%%%%%%%%%%%%%%%%%%
\begin{frame}[allowframebreaks]{\texttt{PRINT\_EIGEN}} \label{PRINT_EIGEN}
\vspace*{-12pt}
\begin{columns}
\column{0.4\linewidth}
\begin{block}{Type}
int
\end{block}

\begin{block}{Default}
0
\end{block}

\column{0.4\linewidth}
\begin{block}{Unit}
No unit
\end{block}

\begin{block}{Example}
\texttt{PRINT\_EIGEN}: 1
\end{block}
\end{columns}

\begin{block}{Description}
Flag for writing eigenvalues and occupations into .eigen file.
\end{block}

\end{frame}
%%%%%%%%%%%%%%%%%%%%%%%%%%%%%%%%%%%%%%%%%%%%%%%%%%%%%%%%%%%%%%%%%%%%%%%%%%%%%%%%%%%%%%%%%%%%%


%%%%%%%%%%%%%%%%%%%%%%%%%%%%%%%%%%%%%%%%%%%%%%%%%%%%%%%%%%%%%%%%%%%%%%%%%%%%%%%%%%%%%%%%%%%%%
\begin{frame}[allowframebreaks]{\texttt{PRINT\_DENSITY}} \label{PRINT_DENSITY}
\vspace*{-12pt}
\begin{columns}
\column{0.4\linewidth}
\begin{block}{Type}
int
\end{block}

\begin{block}{Default}
0
\end{block}

\column{0.4\linewidth}
\begin{block}{Unit}
No unit
\end{block}

\begin{block}{Example}
\texttt{PRINT\_DENSITY}: 1
\end{block}
\end{columns}

\begin{block}{Description}
Flag for writing electron density into cube format. For spin-unpolarized calculation, electron density is printed into .dens file. For collinear spin calculation, total, spin-up and spin-down electron density are printed into .dens, .densUp and .densDwn file, respectively.
\end{block}

\end{frame}
%%%%%%%%%%%%%%%%%%%%%%%%%%%%%%%%%%%%%%%%%%%%%%%%%%%%%%%%%%%%%%%%%%%%%%%%%%%%%%%%%%%%%%%%%%%%%



%%%%%%%%%%%%%%%%%%%%%%%%%%%%%%%%%%%%%%%%%%%%%%%%%%%%%%%%%%%%%%%%%%%%%%%%%%%%%%%%%%%%%%%%%%%%%
\begin{frame}[allowframebreaks]{\texttt{PRINT\_ORBITAL}} \label{PRINT_ORBITAL}
\vspace*{-12pt}
\begin{columns}
\column{0.4\linewidth}
\begin{block}{Type}
int
\end{block}

\begin{block}{Default}
0
\end{block}

\column{0.4\linewidth}
\begin{block}{Unit}
No unit
\end{block}

\begin{block}{Example}
\texttt{PRINT\_ORBITAL}: 1
\end{block}
\end{columns}

\begin{block}{Description}
Flag for writing Kohn-Sham orbitals into a binary file. 
\end{block}

\begin{block}{Remark}
It consists of headers with system information and the orbitals. 
First define a few variables and their types. 
\begin{table}[]
\begin{tabular}{|m{1.8cm}|m{2cm}|m{6.3cm}|}
\hline
name         & type, length       & description                                                                                                   \\ \hline
Nx Ny Nz     & int, 1             & Number of FD nodes in x,y,z directions                                                                        \\ \hline
Nd           & int, 1             & Total number of FD nodes                                                                                      \\ \hline
dx, dy,dz    & double, 1         & mesh size in x,y,z directions                                                                                 \\ \hline
dV           & double, 1          & unit Volume                                                                                                   \\ \hline
Nspinor      & int, 2             & Number of spinor in orbitals                                                                                  \\ \hline
isReal       & int, 1             & Flag for orbitals being real or complex                                                                       \\ \hline
nspin        & int, 1             & Number of spin channel printed                                                                                \\ \hline
nkpt         & int, 1             & Number of k-point printed                                                                                     \\ \hline
\end{tabular}
\end{table}
\end{block}

\begin{block}{Remark -- cont.}
\begin{table}[]
\begin{tabular}{|m{1.7cm}|m{3.2cm}|m{5cm}|}
\hline
nband        & int, 1             & Number of bands printed                                                                                        \\ \hline
name         & type, length       & description                                                                                                   \\ \hline
spin\_index  & int, 1             & spin index of specific orbital                                                                                \\ \hline
kpt\_index   & int, 1             & k-point index of specific orbital                                                                             \\ \hline
kpt\_vec     & double, 3          & k-point in reduced coordinates                                                                                \\ \hline
band\_index  & int, 1             & band index of specific orbital                                                                                \\ \hline
psi\_real    & double, Nd         & real Kohn-Sham orbitals                                                                                       \\ \hline
psi\_complex & double complex, Nd & complex Kohn-Sham orbitals                                                                                    \\ \hline
\end{tabular}
\end{table}
The header is organized as: Nx, Ny, Nz, Nd, dx, dy, dz, dV, Nspinor, isReal, nspin, nkpt, nband, followed by the data for Kohn-sham orbital. Below is the pseudo-code to read orbitals after reading variables in headers. 
\end{block}

\begin{block}{Remark -- cont.}
\begin{algorithm}[H]
\begin{algorithmic}
\For{ispin = 1:nspin}
\For{ikpt = 1:nkpt}
\For{iband = 1:nband}
    \State spin\_index, kpt\_index, kpt\_vec, band\_index
    \If{isReal == 1}
        \State psi\_real
    \Else
        \State psi\_complex
    \EndIf
\EndFor
\EndFor
\EndFor
\end{algorithmic}
\end{algorithm}
\end{block}

\end{frame}
%%%%%%%%%%%%%%%%%%%%%%%%%%%%%%%%%%%%%%%%%%%%%%%%%%%%%%%%%%%%%%%%%%%%%%%%%%%%%%%%%%%%%%%%%%%%%



%%%%%%%%%%%%%%%%%%%%%%%%%%%%%%%%%%%%%%%%%%%%%%%%%%%%%%%%%%%%%%%%%%%%%%%%%%%%%%%%%%%%%%%%%%%%%
\begin{frame}[allowframebreaks]{\texttt{PRINT\_ENERGY\_DENSITY}} \label{PRINT_ENERGY_DENSITY}
\vspace*{-12pt}
\begin{columns}
\column{0.4\linewidth}
\begin{block}{Type}
int
\end{block}

\begin{block}{Default}
0
\end{block}

\column{0.4\linewidth}
\begin{block}{Unit}
No unit
\end{block}

\begin{block}{Example}
\texttt{PRINT\_ENERGY\_DENSITY}: 1
\end{block}
\end{columns}

\begin{block}{Description}
Flag for writing a few energy densities into cube format. Currently, only kinetic energy density, exchange correlation energy density (without exact exchange contribution) and exact exchange energy density (if any) are implemented. 
\end{block}

\begin{block}{Remark}
For spin-unpolarized calculation, kinetic energy density is written into .kedens, exchange correlation energy density is written into .xcedens, and exact exchange energy density is written into .exxedens. For collinear spin calculation, total, spin-up, spin-down kinetic energy density are written into .kedens, kedensUp, kedensDwn files, total, spin-up, spin-down exact exchange energy density are writted into .exxedens, .exxedensUp, .exxedensDwn files, respectively. 
\end{block}
\end{frame}
%%%%%%%%%%%%%%%%%%%%%%%%%%%%%%%%%%%%%%%%%%%%%%%%%%%%%%%%%%%%%%%%%%%%%%%%%%%%%%%%%%%%%%%%%%%%%

