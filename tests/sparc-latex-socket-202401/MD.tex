%%%%%%%%%%%%%%%%%%%%%%%%%%%%%%%%%%%%%%%%%%%%%%%%%%%%%%%%%%%%%%%%%%%%%%%%%%%%%%%%%%%%%%%%%%%%%
\begin{frame}[allowframebreaks,c]{} \label{MD}

\begin{center}
\Huge \textbf{QMD}
\end{center}

\end{frame}
%%%%%%%%%%%%%%%%%%%%%%%%%%%%%%%%%%%%%%%%%%%%%%%%%%%%%%%%%%%%%%%%%%%%%%%%%%%%%%%%%%%%%%%%%%%%%





%%%%%%%%%%%%%%%%%%%%%%%%%%%%%%%%%%%%%%%%%%%%%%%%%%%%%%%%%%%%%%%%%%%%%%%%%%%%%%%%%%%%%%%%%%%%%
\begin{frame}[allowframebreaks]{\texttt{MD\_FLAG}} \label{MD_FLAG}
\vspace*{-12pt}
\begin{columns}
\column{0.4\linewidth}
\begin{block}{Type}
Integer
\end{block}

\begin{block}{Default}
0
\end{block}

\column{0.4\linewidth}
\begin{block}{Unit}
No unit
\end{block}

\begin{block}{Example}
\texttt{MD\_FLAG}: 1
\end{block}
\end{columns}

\begin{block}{Description}
QMD simulations are performed if the flag is set to 1.
\end{block}

\begin{block}{Remark}
\hyperlink{MD_FLAG}{\texttt{MD\_FLAG}} and \hyperlink{RELAX_FLAG}{\texttt{RELAX\_FLAG}} both cannot be set to a value greater than 0. 
\end{block}

\end{frame}
%%%%%%%%%%%%%%%%%%%%%%%%%%%%%%%%%%%%%%%%%%%%%%%%%%%%%%%%%%%%%%%%%%%%%%%%%%%%%%%%%%%%%%%%%%%%%



%%%%%%%%%%%%%%%%%%%%%%%%%%%%%%%%%%%%%%%%%%%%%%%%%%%%%%%%%%%%%%%%%%%%%%%%%%%%%%%%%%%%%%%%%%%%%
\begin{frame}[allowframebreaks]{\texttt{MD\_METHOD}} \label{MD_METHOD}
\vspace*{-12pt}
\begin{columns}
\column{0.4\linewidth}
\begin{block}{Type}
String
\end{block}

\begin{block}{Default}
NVT\_NH
\end{block}

\column{0.4\linewidth}
\begin{block}{Unit}
No unit
\end{block}

\begin{block}{Example}
\texttt{MD\_METHOD}: NVE
\end{block}
\end{columns}

\begin{block}{Description}
Type of QMD to be performed. Currently, NVE (microcanonical ensemble), NVT\_NH (canonical ensemble with Nose-Hoover thermostat), NVK\_G (isokinetic ensemble with Gaussian thermostat), NPT\_NH (isothermal-isobaric ensemble with Nose-Hoover thermostat) and NPT\_NP (isothermal-isobaric ensemble with Nose-Poincare thermostat) are supported.
\end{block}

\end{frame}
%%%%%%%%%%%%%%%%%%%%%%%%%%%%%%%%%%%%%%%%%%%%%%%%%%%%%%%%%%%%%%%%%%%%%%%%%%%%%%%%%%%%%%%%%%%%%



%%%%%%%%%%%%%%%%%%%%%%%%%%%%%%%%%%%%%%%%%%%%%%%%%%%%%%%%%%%%%%%%%%%%%%%%%%%%%%%%%%%%%%%%%%%%%
\begin{frame}[allowframebreaks]{\texttt{MD\_NSTEP}} \label{MD_NSTEP}
\vspace*{-12pt}
\begin{columns}
\column{0.4\linewidth}
\begin{block}{Type}
Integer
\end{block}

\begin{block}{Default}
1e7
\end{block}

\column{0.4\linewidth}
\begin{block}{Unit}
No unit
\end{block}

\begin{block}{Example}
\texttt{MD\_NSTEP}: 100
\end{block}
\end{columns}

\begin{block}{Description}
Specifies the number of QMD steps.
\end{block}

\begin{block}{Remark}
If MD\_NSTEP $= N$, the QMD runs from $0$ to $(N-1) \times$ \hyperlink{MD_TIMESTEP}{\texttt{MD\_TIMESTEP}} fs.  
\end{block}

\end{frame}
%%%%%%%%%%%%%%%%%%%%%%%%%%%%%%%%%%%%%%%%%%%%%%%%%%%%%%%%%%%%%%%%%%%%%%%%%%%%%%%%%%%%%%%%%%%%%


%%%%%%%%%%%%%%%%%%%%%%%%%%%%%%%%%%%%%%%%%%%%%%%%%%%%%%%%%%%%%%%%%%%%%%%%%%%%%%%%%%%%%%%%%%%%%
\begin{frame}[allowframebreaks]{\texttt{MD\_TIMESTEP}} \label{MD_TIMESTEP}
\vspace*{-12pt}
\begin{columns}
\column{0.4\linewidth}
\begin{block}{Type}
Double
\end{block}

\begin{block}{Default}
1
\end{block}

\column{0.4\linewidth}
\begin{block}{Unit}
Femtosecond
\end{block}

\begin{block}{Example}
\texttt{MD\_TIMESTEP}: 0.1
\end{block}
\end{columns}

\begin{block}{Description}
QMD time step. 
\end{block}

\begin{block}{Remark}
Total QMD time is given by:  \hyperlink{MD_TIMESTEP}{\texttt{MD\_TIMESTEP}} $\times$ (\hyperlink{MD_NSTEP}{\texttt{MD\_NSTEP}} - 1).
\end{block}

\end{frame}
%%%%%%%%%%%%%%%%%%%%%%%%%%%%%%%%%%%%%%%%%%%%%%%%%%%%%%%%%%%%%%%%%%%%%%%%%%%%%%%%%%%%%%%%%%%%%


%%%%%%%%%%%%%%%%%%%%%%%%%%%%%%%%%%%%%%%%%%%%%%%%%%%%%%%%%%%%%%%%%%%%%%%%%%%%%%%%%%%%%%%%%%%%%
\begin{frame}[allowframebreaks]{\texttt{ION\_TEMP}} \label{ION_TEMP}
\vspace*{-12pt}
\begin{columns}
\column{0.4\linewidth}
\begin{block}{Type}
Double
\end{block}

\begin{block}{Default}
No Default
\end{block}

\column{0.4\linewidth}
\begin{block}{Unit}
Kelvin
\end{block}

\begin{block}{Example}
\texttt{ION\_TEMP}: 315
\end{block}
\end{columns}

\begin{block}{Description}
Starting ionic temperature in QMD, used to generate initial velocity distribution.
\end{block}

\begin{block}{Remark}
Must be specified if \hyperlink{MD_FLAG}{\texttt{MD\_FLAG}} is set to $1$. It is also the target temperature in \hyperlink{MD_METHOD}{\texttt{MD\_METHOD}} NPT\_NH and NPT\_NP.
\end{block}

\end{frame}
%%%%%%%%%%%%%%%%%%%%%%%%%%%%%%%%%%%%%%%%%%%%%%%%%%%%%%%%%%%%%%%%%%%%%%%%%%%%%%%%%%%%%%%%%%%%%



%%%%%%%%%%%%%%%%%%%%%%%%%%%%%%%%%%%%%%%%%%%%%%%%%%%%%%%%%%%%%%%%%%%%%%%%%%%%%%%%%%%%%%%%%%%%%
\begin{frame}[allowframebreaks]{\texttt{ION\_TEMP\_END}} \label{ION_TEMP_END}
\vspace*{-12pt}
\begin{columns}
\column{0.4\linewidth}
\begin{block}{Type}
Double
\end{block}

\begin{block}{Default}
ION\_TEMP
\end{block}

\column{0.4\linewidth}
\begin{block}{Unit}
Kelvin
\end{block}

\begin{block}{Example}
\texttt{ION\_TEMP\_END}: 100
\end{block}
\end{columns}

\begin{block}{Description}
Specifies the final temperature of the thermostat. Thermostat temperature is varied linearly from \hyperlink{ION_TEMP}{\texttt{ION\_TEMP}} to \hyperlink{ION_TEMP_END}{\texttt{ION\_TEMP\_END}} with respect to time.
\end{block}

\begin{block}{Remark}
Available for NVT\_NH quantum molecular dynamics only. Not supported in NPT\_NH and NPT\_NP.
\end{block}

\end{frame}
%%%%%%%%%%%%%%%%%%%%%%%%%%%%%%%%%%%%%%%%%%%%%%%%%%%%%%%%%%%%%%%%%%%%%%%%%%%%%%%%%%%%%%%%%%%%%

%%%%%%%%%%%%%%%%%%%%%%%%%%%%%%%%%%%%%%%%%%%%%%%%%%%%%%%%%%%%%%%%%%%%%%%%%%%%%%%%%%%%%%%%%%%%%
% \begin{frame}[allowframebreaks]{\texttt{ION\_ELEC\_EQT}} \label{ION_ELEC_EQT}
% \vspace*{-12pt}
% \begin{columns}
% \column{0.4\linewidth}
% \begin{block}{Type}
% Integer
% \end{block}

% \begin{block}{Default}
% 1
% \end{block}

% \column{0.4\linewidth}
% \begin{block}{Unit}
% No unit
% \end{block}

% \begin{block}{Example}
% \texttt{ION\_ELEC\_EQT}: 0
% \end{block}
% \end{columns}

% \begin{block}{Description}
% Flag that determines whether the \hyperlink{ELEC_TEMP}{\texttt{ELEC\_TEMP}} will be set equal to \hyperlink{ION_TEMP}{\texttt{ION\_TEMP}} during MD.
% \end{block}

% \begin{block}{Remark}
% If the flag is set to 0, the values of \hyperlink{ELEC_TEMP}{\texttt{ELEC\_TEMP}} and \hyperlink{ION_TEMP}{\texttt{ION\_TEMP}} need to be identical.  
% \end{block}

% \end{frame}
%%%%%%%%%%%%%%%%%%%%%%%%%%%%%%%%%%%%%%%%%%%%%%%%%%%%%%%%%%%%%%%%%%%%%%%%%%%%%%%%%%%%%%%%%%%%%



%%%%%%%%%%%%%%%%%%%%%%%%%%%%%%%%%%%%%%%%%%%%%%%%%%%%%%%%%%%%%%%%%%%%%%%%%%%%%%%%%%%%%%%%%%%%%
\begin{frame}[allowframebreaks]{\texttt{ION\_VEL\_DSTR}} \label{ION_VEL_DSTR}
\vspace*{-12pt}
\begin{columns}
\column{0.4\linewidth}
\begin{block}{Type}
Integer
\end{block}

\begin{block}{Default}
2
\end{block}

\column{0.4\linewidth}
\begin{block}{Unit}
No unit
\end{block}

\begin{block}{Example}
\texttt{ION\_VEL\_DSTR}: 1
\end{block}
\end{columns}

\begin{block}{Description}
Specifies the type of distribution for the initial velocity of atoms based on their initial temperature. Choose $1$ for uniform velocity distribution and $2$ for Maxwell-Boltzmann distribution.
\end{block}

\begin{block}{Remark}
Currently, the code supports only two options for the variable.
\end{block}

\end{frame}
%%%%%%%%%%%%%%%%%%%%%%%%%%%%%%%%%%%%%%%%%%%%%%%%%%%%%%%%%%%%%%%%%%%%%%%%%%%%%%%%%%%%%%%%%%%%%


%%%%%%%%%%%%%%%%%%%%%%%%%%%%%%%%%%%%%%%%%%%%%%%%%%%%%%%%%%%%%%%%%%%%%%%%%%%%%%%%%%%%%%%%%%%%%
\begin{frame}[allowframebreaks]{\texttt{ION\_VEL\_DSTR\_RAND}} \label{ION_VEL_DSTR_RAND}
\vspace*{-12pt}
\begin{columns}
\column{0.4\linewidth}
\begin{block}{Type}
Integer
\end{block}

\begin{block}{Default}
0
\end{block}

\column{0.4\linewidth}
\begin{block}{Unit}
No unit
\end{block}

\begin{block}{Example}
\texttt{ION\_VEL\_DSTR\_RAND}: 1
\end{block}
\end{columns}

\begin{block}{Description}
Flag to reseed the initial velocities of atoms in a QMD simulation. Set this flag to $1$ to shuffle (change the random seed for) the initial velocities for different runs. Set this flag to $0$ to maintain the same initial velocities.
\end{block}

\begin{block}{Remark}
This option is convenient for parallel statistics calculations.
\end{block}

\end{frame}
%%%%%%%%%%%%%%%%%%%%%%%%%%%%%%%%%%%%%%%%%%%%%%%%%%%%%%%%%%%%%%%%%%%%%%%%%%%%%%%%%%%%%%%%%%%%%



%%%%%%%%%%%%%%%%%%%%%%%%%%%%%%%%%%%%%%%%%%%%%%%%%%%%%%%%%%%%%%%%%%%%%%%%%%%%%%%%%%%%%%%%%%%%%
\begin{frame}[allowframebreaks]{\texttt{QMASS}} \label{QMASS}
\vspace*{-12pt}
\begin{columns}
\column{0.4\linewidth}
\begin{block}{Type}
Double
\end{block}

\begin{block}{Default}
1653.654933459720
\end{block}

\column{0.4\linewidth}
\begin{block}{Unit}
atomic unit
\end{block}

\begin{block}{Example}
\texttt{QMASS}: 100000
\end{block}
\end{columns}

\begin{block}{Description}
Gives the inertia factor for Nose Hoover thermostat.
\end{block}

\begin{block}{Remark}
Applicable to NVT\_NH \hyperlink{MD_METHOD}{\texttt{MD\_METHOD}} only.
\end{block}

\end{frame}
%%%%%%%%%%%%%%%%%%%%%%%%%%%%%%%%%%%%%%%%%%%%%%%%%%%%%%%%%%%%%%%%%%%%%%%%%%%%%%%%%%%%%%%%%%%%%



%%%%%%%%%%%%%%%%%%%%%%%%%%%%%%%%%%%%%%%%%%%%%%%%%%%%%%%%%%%%%%%%%%%%%%%%%%%%%%%%%%%%%%%%%%%%%
\begin{frame}[allowframebreaks]{\texttt{NPT\_NH\_QMASS}} \label{NPT_NH_QMASS}
\vspace*{-12pt}
\begin{columns}
\column{0.4\linewidth}
\begin{block}{Type}
1st number int; others double
\end{block}

\begin{block}{Default}
No default value
\end{block}

\column{0.4\linewidth}
\begin{block}{Unit}
atomic unit
\end{block}

\begin{block}{Example}
\texttt{NPT\_NH\_QMASS}: 2\;700.0\;700.0
\end{block}
\end{columns}

\begin{block}{Description}
Gives the amount (first number) and inertia masses (others) of thermostats in NPT\_NH. The maximum amount of thermostat variables of the Nose-Hoover chain is 60
\end{block}

\begin{block}{Remark}
Applicable to NPT\_NH \hyperlink{MD_METHOD}{\texttt{MD\_METHOD}} only.
Program will exit if NPT\_NH is selected but NPT\_NH\_QMASS is not input
\end{block}

\end{frame}
%%%%%%%%%%%%%%%%%%%%%%%%%%%%%%%%%%%%%%%%%%%%%%%%%%%%%%%%%%%%%%%%%%%%%%%%%%%%%%%%%%%%%%%%%%%%%



%%%%%%%%%%%%%%%%%%%%%%%%%%%%%%%%%%%%%%%%%%%%%%%%%%%%%%%%%%%%%%%%%%%%%%%%%%%%%%%%%%%%%%%%%%%%%
\begin{frame}[allowframebreaks]{\texttt{NPT\_NH\_BMASS}} \label{NPT_NH_BMASS}
\vspace*{-12pt}
\begin{columns}
\column{0.4\linewidth}
\begin{block}{Type}
Double
\end{block}

\begin{block}{Default}
No default value
\end{block}

\column{0.4\linewidth}
\begin{block}{Unit}
atomic unit
\end{block}

\begin{block}{Example}
\texttt{NPT\_NH\_BMASS}: 5000
\end{block}
\end{columns}

\begin{block}{Description}
Gives the inertia mass for the barostat variable in NPT\_NH.
\end{block}

\begin{block}{Remark}
Applicable to NPT\_NH \hyperlink{MD_METHOD}{\texttt{MD\_METHOD}} only.
Program will exit if NPT\_NH is selected but NPT\_NH\_BMASS is not input
\end{block}

\end{frame}
%%%%%%%%%%%%%%%%%%%%%%%%%%%%%%%%%%%%%%%%%%%%%%%%%%%%%%%%%%%%%%%%%%%%%%%%%%%%%%%%%%%%%%%%%%%%%



%%%%%%%%%%%%%%%%%%%%%%%%%%%%%%%%%%%%%%%%%%%%%%%%%%%%%%%%%%%%%%%%%%%%%%%%%%%%%%%%%%%%%%%%%%%%%
\begin{frame}[allowframebreaks]{\texttt{NPT\_NP\_QMASS}} \label{NPT_NP_QMASS}
\vspace*{-12pt}
\begin{columns}
\column{0.4\linewidth}
\begin{block}{Type}
Double
\end{block}

\begin{block}{Default}
No default value
\end{block}

\column{0.4\linewidth}
\begin{block}{Unit}
atomic unit
\end{block}

\begin{block}{Example}
\texttt{NPT\_NP\_QMASS}: 100
\end{block}
\end{columns}

\begin{block}{Description}
Gives the inertia mass for the thermostat variable in NPT\_NP.
\end{block}

\begin{block}{Remark}
Applicable to NPT\_NP \hyperlink{MD_METHOD}{\texttt{MD\_METHOD}} only.
Program will exit if NPT\_NP is selected but NPT\_NP\_BMASS is not input
\end{block}

\end{frame}
%%%%%%%%%%%%%%%%%%%%%%%%%%%%%%%%%%%%%%%%%%%%%%%%%%%%%%%%%%%%%%%%%%%%%%%%%%%%%%%%%%%%%%%%%%%%%



%%%%%%%%%%%%%%%%%%%%%%%%%%%%%%%%%%%%%%%%%%%%%%%%%%%%%%%%%%%%%%%%%%%%%%%%%%%%%%%%%%%%%%%%%%%%%
\begin{frame}[allowframebreaks]{\texttt{NPT\_NP\_BMASS}} \label{NPT_NP_BMASS}
\vspace*{-12pt}
\begin{columns}
\column{0.4\linewidth}
\begin{block}{Type}
Double
\end{block}

\begin{block}{Default}
No default value
\end{block}

\column{0.4\linewidth}
\begin{block}{Unit}
atomic unit
\end{block}

\begin{block}{Example}
\texttt{NPT\_NP\_BMASS}: 20
\end{block}
\end{columns}

\begin{block}{Description}
Gives the inertia mass for the barostat variable in NPT\_NP.
\end{block}

\begin{block}{Remark}
Applicable to NPT\_NP \hyperlink{MD_METHOD}{\texttt{MD\_METHOD}} only.
Program will exit if NPT\_NP is selected but NPT\_NP\_BMASS is not input
\end{block}

\end{frame}
%%%%%%%%%%%%%%%%%%%%%%%%%%%%%%%%%%%%%%%%%%%%%%%%%%%%%%%%%%%%%%%%%%%%%%%%%%%%%%%%%%%%%%%%%%%%%



%%%%%%%%%%%%%%%%%%%%%%%%%%%%%%%%%%%%%%%%%%%%%%%%%%%%%%%%%%%%%%%%%%%%%%%%%%%%%%%%%%%%%%%%%%%%%
\begin{frame}[allowframebreaks]{\texttt{NPT\_SCALE\_VECS}} \label{NPT_SCALE_VECS}
\vspace*{-12pt}
\begin{columns}
\column{0.4\linewidth}
\begin{block}{Type}
Int
\end{block}

\begin{block}{Default}
1 2 3
\end{block}

\column{0.4\linewidth}
\begin{block}{Unit}
No unit
\end{block}

\begin{block}{Example}
\texttt{NPT\_SCALE\_VECS}: 2
\end{block}
\end{columns}

\begin{block}{Description}
Specify which lattice vectors can be rescaled in NPT\_NH and NPT\_NP. The cell will only expand or shrink in the specified directions.

Rescaled vectors can be specified for orthogonal systems if NPT\_NP thermostat is used.
\end{block}

\begin{block}{Remark}
Only three numbers 1, 2 and 3 can be accepted. For example, if ``2 3'' is the input, the cell will only expand or shrink in the directions of lattice vector 2 and lattice vector 3.

If it is set in NPT\_NH, the expansion or shrinkage on designated lattice vector will try to keep the total pressure to oscillate near the target pressure.

If it is set in NPT\_NP, the expansion or shrinkage on designated lattice vector will only try to keep the normal stress at their direction to oscillate near the target pressure.
\end{block}

\end{frame}
%%%%%%%%%%%%%%%%%%%%%%%%%%%%%%%%%%%%%%%%%%%%%%%%%%%%%%%%%%%%%%%%%%%%%%%%%%%%%%%%%%%%%%%%%%%%%



%%%%%%%%%%%%%%%%%%%%%%%%%%%%%%%%%%%%%%%%%%%%%%%%%%%%%%%%%%%%%%%%%%%%%%%%%%%%%%%%%%%%%%%%%%%%%
\begin{frame}[allowframebreaks]{\texttt{NPT\_SCALE\_CONSTRAINTS}} \label{NPT_SCALE_CONSTRAINTS}
\vspace*{-12pt}
\begin{columns}
\column{0.4\linewidth}
\begin{block}{Type}
Double
\end{block}

\begin{block}{Default}
none
\end{block}

\column{0.4\linewidth}
\begin{block}{Unit}
No unit
\end{block}

\begin{block}{Example}
\texttt{NPT\_SCALE\_CONSTRAINTS}: 12
\end{block}
\end{columns}

\begin{block}{Description}
Set the scale constraint for lattice vectors in NPT\_NP. The length ratio between the designated lattice vector keeps constant in NPT\_NP thermostat. For example, if ``12'' is set, then the length ratio between 1st and 2nd lattice vectors will keep constant.
\end{block}

\begin{block}{Remark}
Applicable to orthogonal system using NPT\_NP \hyperlink{MD_METHOD}{\texttt{MD\_METHOD}} only.

There are 4 types of available constraints. ``12'' or ``21''; ``13'' or ``31''; ``23'' or ``32''; ``123'' or ``132'' or ``213'' or ``231'' or ``312'' or ``321''.
\end{block}

\end{frame}
%%%%%%%%%%%%%%%%%%%%%%%%%%%%%%%%%%%%%%%%%%%%%%%%%%%%%%%%%%%%%%%%%%%%%%%%%%%%%%%%%%%%%%%%%%%%%



%%%%%%%%%%%%%%%%%%%%%%%%%%%%%%%%%%%%%%%%%%%%%%%%%%%%%%%%%%%%%%%%%%%%%%%%%%%%%%%%%%%%%%%%%%%%%
\begin{frame}[allowframebreaks]{\texttt{TARGET\_PRESSURE}} \label{TARGET_PRESSURE}
\vspace*{-12pt}
\begin{columns}
\column{0.4\linewidth}
\begin{block}{Type}
Double
\end{block}

\begin{block}{Default}
0.0
\end{block}

\column{0.4\linewidth}
\begin{block}{Unit}
GPa
\end{block}

\begin{block}{Example}
\texttt{TARGET\_PRESSURE}: 40.9611
\end{block}
\end{columns}

\begin{block}{Description}
Gives the outer pressure in NPT\_NH and NPT\_NP.
\end{block}

\begin{block}{Remark}
Applicable to NPT\_NH and NPT\_NP \hyperlink{MD_METHOD}{\texttt{MD\_METHOD}} only.
\end{block}

\end{frame}
%%%%%%%%%%%%%%%%%%%%%%%%%%%%%%%%%%%%%%%%%%%%%%%%%%%%%%%%%%%%%%%%%%%%%%%%%%%%%%%%%%%%%%%%%%%%%



%%%%%%%%%%%%%%%%%%%%%%%%%%%%%%%%%%%%%%%%%%%%%%%%%%%%%%%%%%%%%%%%%%%%%%%%%%%%%%%%%%%%%%%%%%%%%
\begin{frame}[allowframebreaks]{\texttt{RESTART\_FLAG}} \label{RESTART_FLAG}
\vspace*{-12pt}
\begin{columns}
\column{0.4\linewidth}
\begin{block}{Type}
Integer
\end{block}

\begin{block}{Default}
0
\end{block}

\column{0.4\linewidth}
\begin{block}{Unit}
No unit
\end{block}

\begin{block}{Example}
\texttt{RESTART\_FLAG}: 0
\end{block}
\end{columns}

\begin{block}{Description}
Flag for restarting quantum molecular dynamics and structural relaxation. Stores last three histories for quantum molecular dynamics simulations in .restart, .restart-$0$ and .restart-$1$ files, respectively.
\end{block}

\begin{block}{Remark}
Restarts from the previous configuration which is stored in a .restart file. Currently, code provides restart feature for atomic relaxation and QMD only. 
\end{block}

\end{frame}
%%%%%%%%%%%%%%%%%%%%%%%%%%%%%%%%%%%%%%%%%%%%%%%%%%%%%%%%%%%%%%%%%%%%%%%%%%%%%%%%%%%%%%%%%%%%%%%%%%




%%%%%%%%%%%%%%%%%%%%%%%%%%%%%%%%%%%%%%%%%%%%%%%%%%%%%%%%%%%%%%%%%%%%%%%%%%%%%%%%%%%%%%%%%%%%%
\begin{frame}[allowframebreaks]{\texttt{TWTIME}} \label{TWTIME}
\vspace*{-12pt}
\begin{columns}
\column{0.4\linewidth}
\begin{block}{Type}
Double
\end{block}

\begin{block}{Default}
1e9
\end{block}

\column{0.4\linewidth}
\begin{block}{Unit}
min
\end{block}

\begin{block}{Example}
\texttt{TWTIME}: 1000
\end{block}
\end{columns}

\begin{block}{Description}
Gives the upper bound on the wall time for quantum molecular dynamics.
\end{block}

\end{frame}
%%%%%%%%%%%%%%%%%%%%%%%%%%%%%%%%%%%%%%%%%%%%%%%%%%%%%%%%%%%%%%%%%%%%%%%%%%%%%%%%%%%%%%%%%%%%%