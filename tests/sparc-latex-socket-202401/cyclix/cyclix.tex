
%%%%%%%%%%%%%%%%%%%%%%%%%%%%%%%%%%%%%%%%%%%%%%%%%%%%%%%%%%%%%%%%%%%%%%%%%%%%%%%%%%%%%%%%%%%%%
%\begin{frame}[allowframebreaks,c]{} \label{Cyclix}
%
%\begin{center}
%\Huge \textbf{Cyclix}
%\end{center}
%
%\end{frame}
%%%%%%%%%%%%%%%%%%%%%%%%%%%%%%%%%%%%%%%%%%%%%%%%%%%%%%%%%%%%%%%%%%%%%%%%%%%%%%%%%%%%%%%%%%%%%

%%%%%%%%%%%%%%%%%%%%%%%%%%%%%%%%%%%%%%%%%%%%%%%%%%%%%%%%%%%%%%%%%%%%%%%%%%%%%%%%%%%%%%%%%%%%%
\begin{frame}[allowframebreaks]{\texttt{TWIST\_ANGLE}} \label{TWIST_ANGLE}
\vspace*{-12pt}
\begin{columns}
\column{0.4\linewidth}
\begin{block}{Type}
Double
\end{block}

\begin{block}{Default}
0
\end{block}

\column{0.4\linewidth}
\begin{block}{Unit}
rad/Bohr
\end{block}

\begin{block}{Example}
\texttt{TWIST\_ANGLE}: 0.0045
\end{block}
\end{columns}

\begin{block}{Description}
External twist per unit length applied on the nanotube.
\end{block}

\begin{block}{Remark}
If using helical symmetry (D C H), we also have to add the intrinsic twist.
\end{block}

\end{frame}
%%%%%%%%%%%%%%%%%%%%%%%%%%%%%%%%%%%%%%%%%%%%%%%%%%%%%%%%%%%%%%%%%%%%%%%%%%%%%%%%%%%%%%%%%%%%%

%%%%%%%%%%%%%%%%%%%%%%%%%%%%%%%%%%%%%%%%%%%%%%%%%%%%%%%%%%%%%%%%%%%%%%%%%%%%%%%%%%%%%%%%%%%%%
\begin{frame}[allowframebreaks]{\texttt{{BC}}} \label{BC}
\vspace*{-12pt}
\begin{columns}
\column{0.4\linewidth}
\begin{block}{Type}
Character
\end{block}

\begin{block}{Default}
None
\end{block}

\column{0.4\linewidth}
\begin{block}{Unit}
No Unit
\end{block}

\begin{block}{Example}
\texttt{BC}: D C P
\end{block}
\end{columns}

\begin{block}{Description}
A set of three whitespace delimited characters specifying the boundary conditions in the lattice vector directions, respectively. P represents periodic boundary conditions, D represents Dirichlet boundary conditions, C represents cyclic boundary conditions and H represents helical boundary conditions.
\end{block}

\begin{block}{Remark}
\texttt{BC}: D C P and \texttt{BC}: D C H are the cyclix cases where the former uses only cyclic boundary condition and latter uses both cyclic and helical boundary conditions.
\end{block}
\end{frame}
%%%%%%%%%%%%%%%%%%%%%%%%%%%%%%%%%%%%%%%%%%%%%%%%%%%%%%%%%%%%%%%%%%%%%%%%%%%%%%%%%%%%%%%%%%%%%

%%%%%%%%%%%%%%%%%%%%%%%%%%%%%%%%%%%%%%%%%%%%%%%%%%%%%%%%%%%%%%%%%%%%%%%%%%%%%%%%%%%%%%%%%%%%%
\begin{frame}[allowframebreaks]{\texttt{{CELL}}} \label{CELL}
\vspace*{-12pt}
\begin{columns}
\column{0.4\linewidth}
\begin{block}{Type}
Double
\end{block}

\begin{block}{Default}
None
\end{block}

\column{0.4\linewidth}
\begin{block}{Unit}
Bohr
\end{block}

\begin{block}{Example}
\texttt{CELL}: 30.3498 0.2991 3.1359
\end{block}
\end{columns}

\begin{block}{Description}
A set of three whitespace delimited values specifying the cell lengths in the radial, angular and periodic/helical directions, respectively.
\end{block}

\begin{block}{Remark}
Angular direction length is equal to $ 2\pi / \Gamma$, where $ \Gamma $ is the group order in the cyclic direction.
\end{block}
\end{frame}
%%%%%%%%%%%%%%%%%%%%%%%%%%%%%%%%%%%%%%%%%%%%%%%%%%%%%%%%%%%%%%%%%%%%%%%%%%%%%%%%%%%%%%%%%%%%%


%%%%%%%%%%%%%%%%%%%%%%%%%%%%%%%%%%%%%%%%%%%%%%%%%%%%%%%%%%%%%%%%%%%%%%%%%%%%%%%%%%%%%%%%%%%%%
\begin{frame}[allowframebreaks]{\texttt{{COORD}}} \label{COORD}
\vspace*{-12pt}
\begin{columns}
\column{0.4\linewidth}
\begin{block}{Type}
Double
\end{block}

\begin{block}{Default}
None
\end{block}

\column{0.4\linewidth}
\begin{block}{Unit}
Bohr
\end{block}

\begin{block}{Example}
\texttt{COORD}: \\
0 0 0 \\
2.5 2.5 2.5
\end{block}
\end{columns}

\begin{block}{Description}
The Cartesian coordinates of atoms of a \texttt{ATOM\_TYPE} specified before this variable. If the coordinates are outside the fundamental domain (see \hyperlink{CELL}{\texttt{CELL}}) in the periodic and cyclic/helical directions (see \hyperlink{BC}{\texttt{BC}}), it will be automatically mapped back to the domain.
\end{block}

\begin{block}{Remark}
For a system with different types of atoms, one has to specify the coordinates for every \texttt{ATOM\_TYPE}. One can also specify the coordinates of the atoms using \hyperlink{COORD_FRAC}{\texttt{COORD\_FRAC}}.
\end{block}
\end{frame}
%%%%%%%%%%%%%%%%%%%%%%%%%%%%%%%%%%%%%%%%%%%%%%%%%%%%%%%%%%%%%%%%%%%%%%%%%%%%%%%%%%%%%%%%%%%%%


%%%%%%%%%%%%%%%%%%%%%%%%%%%%%%%%%%%%%%%%%%%%%%%%%%%%%%%%%%%%%%%%%%%%%%%%%%%%%%%%%%%%%%%%%%%%%
\begin{frame}[allowframebreaks]{\texttt{{COORD\_FRAC}}} \label{COORD_FRAC}
\vspace*{-12pt}
\begin{columns}
\column{0.4\linewidth}
\begin{block}{Type}
Double
\end{block}

\begin{block}{Default}
None
\end{block}

\column{0.4\linewidth}
\begin{block}{Unit}
None
\end{block}

\begin{block}{Example}
\texttt{COORD\_FRAC}: \\
0 0 0 \\
2.1 0.5 0.5
\end{block}
\end{columns}

\begin{block}{Description}
The fractional coordinates of atoms of a \texttt{ATOM\_TYPE} specified before this variable. $\texttt{{COORD\_FRAC}}(i , j) \times \hyperlink{CELL}{\texttt{CELL}}(j), (j = 1, 2, 3)$ gives the coordinate of the $i^{th}$ atom along the $j^{th}$ direction. $j = 1$ is along radial, $j = 2$ is along angular and $j = 3$ is along periodic/helical direction.
\end{block}

\begin{block}{Remark}
For a system with different types of atoms, one has to specify the coordinates for every \texttt{ATOM\_TYPE}. One can also specify the coordinates of the atoms using \hyperlink{COORD}{\texttt{COORD}}.
\end{block}
\end{frame}
%%%%%%%%%%%%%%%%%%%%%%%%%%%%%%%%%%%%%%%%%%%%%%%%%%%%%%%%%%%%%%%%%%%%%%%%%%%%%%%%%%%%%%%%%%%%%

%%%%%%%%%%%%%%%%%%%%%%%%%%%%%%%%%%%%%%%%%%%%%%%%%%%%%%%%%%%%%%%%%%%%%%%%%%%%%%%%%%%%%%%%%%%%%
\begin{frame}[allowframebreaks]{\texttt{{EXCHANGE\_CORRELATION}}} \label{EXCHANGE_CORRELATION}
\vspace*{-12pt}
\begin{columns}
\column{0.4\linewidth}
\begin{block}{Type}
String
\end{block}

\begin{block}{Default}
No Default
\end{block}

\column{0.4\linewidth}
\begin{block}{Unit}
No unit
\end{block}

\begin{block}{Example}
\texttt{EXCHANGE\_CORRELATION: LDA\_PW}
\end{block}
\end{columns}

\begin{block}{Description}
Choice of exchange-correlation functional. Options are \texttt{LDA\_PW} (Perdew-Wang LDA), \texttt{LDA\_PZ} (Purdew-Zunger LDA) and \texttt{GGA\_PBE} (PBE GGA)
\end{block}

\begin{block}{Remark}
For spin-polarized calculation (\texttt{{SPIN\_TYP}} = 1),  \texttt{LDA\_PW} (Perdew-Wang LDA) and \texttt{GGA\_PBE} (PBE GGA) are available.
\end{block}


\end{frame}
%%%%%%%%%%%%%%%%%%%%%%%%%%%%%%%%%%%%%%%%%%%%%%%%%%%%%%%%%%%%%%%%%%%%%%%%%%%%%%%%%%%%%%%%%%%%%


\begin{frame}[allowframebreaks]{\texttt{KPOINT\_GRID}} \label{KPOINT_GRID}
\vspace*{-12pt}
\begin{columns}
\column{0.4\linewidth}
\begin{block}{Type}
Integer array
\end{block}

\begin{block}{Default}
 1 1 1
\end{block}

\column{0.4\linewidth}
\begin{block}{Unit}
No unit
\end{block}

\begin{block}{Example}
\texttt{KPOINT\_GRID}: 1 5 10
\end{block}
\end{columns}

\begin{block}{Description}
Number of k-points in each direction of the Monkhorst-Pack grid for Brillouin zone integration. In SPARC-cyclix, we don't have any k-points along the radial direction as it is vaccuum. The angular k-point is reffered as $\nu$ point and the periodic/helical k-point is called $\eta$ point.
\end{block}

\begin{block}{Remark}
$\nu$ point should be an integer factor of the group order $\Gamma$.
\end{block}

\end{frame}



%%%%%%%%%%%%%%%%%%%%%%%%%%%%%%%%%%%%%%%%%%%%%%%%%%%%%%%%%%%%%%%%%%%%%%%%%%%%%%%%%%%%%%%%%%%%%
%\begin{frame}[allowframebreaks]{\texttt{{SPIN\_TYP}}} \label{SPIN_TYP}
%\vspace*{-12pt}
%\begin{columns}
%\column{0.4\linewidth}
%\begin{block}{Type}
%Integer
%\end{block}
%
%\begin{block}{Default}
%0
%\end{block}
%
%\column{0.4\linewidth}
%\begin{block}{Unit}
%No unit
%\end{block}
%
%\begin{block}{Example}
%\texttt{SPIN\_TYP}: 1
%\end{block}
%\end{columns}
%
%\begin{block}{Description}
%\texttt{{SPIN\_TYP}}: 0 performs spin unpolarized calculation. \\
%\texttt{{SPIN\_TYP}}: 1 performs unconstrained collinear spin-polarized calculation.
%\end{block}
%
%\begin{block}{Remark}
%\texttt{{SPIN\_TYP}} can only take values 0 and 1.
%\end{block}
%
%
%\end{frame}
%%%%%%%%%%%%%%%%%%%%%%%%%%%%%%%%%%%%%%%%%%%%%%%%%%%%%%%%%%%%%%%%%%%%%%%%%%%%%%%%%%%%%%%%%%%%%%
%
%
%%%%%%%%%%%%%%%%%%%%%%%%%%%%%%%%%%%%%%%%%%%%%%%%%%%%%%%%%%%%%%%%%%%%%%%%%%%%%%%%%%%%%%%%%%%%%%
%\begin{frame}[allowframebreaks]{\texttt{{SPIN}}} \label{SPIN}
%\vspace*{-12pt}
%\begin{columns}
%\column{0.4\linewidth}
%\begin{block}{Type}
%Double
%\end{block}
%
%\begin{block}{Default}
%0.0
%\end{block}
%
%\column{0.4\linewidth}
%\begin{block}{Unit}
%No unit
%\end{block}
%
%\begin{block}{Example}
%\texttt{SPIN}: \\
%1.0 \\
%-1.0
%\end{block}
%\end{columns}
%
%\begin{block}{Description}
%Specifies the net initial spin on each atom for a spin-polarized calculation.
%\end{block}

%\end{frame}
%%%%%%%%%%%%%%%%%%%%%%%%%%%%%%%%%%%%%%%%%%%%%%%%%%%%%%%%%%%%%%%%%%%%%%%%%%%%%%%%%%%%%%%%%%%%%



